\section{Introduction}
Internet of Things devices get rapidly more and more integrated in our daily life \cite{Yu2017}. In fact the number of IoT devices are expected to almost triple until 2030 accordingly to a statistic from Statista in cooperation with Transform Insights \cite{StatisticTransformaInsights} \cite{StatisticTransformaInsightsByUseCase}. Besides the growth, security remains the greatest obstacle to IoT. Methods to secure all devices from attacks to protect the privacy of individuals or overall protect the public safety are still not well established yet \cite{Bettayeb2019}. A series of partly critical security vulnerabilities like Ripple20 and Amnesia:33 show that even commonly used libraries have critical vulnerabilities which can be exploited by attackers. IoT devices are affected by these two vulnerability collections to a large extent. The problem here is the update philosophy of the vendors. Many cheap IoT products are usually no longer provided with any updates at all \cite{OliviavonWesternhagen} \cite{Schmidt}.

In this paper, we investigate different attack vectors on the firmware update process and what security counter measures can protect the firmware update process. In particular, the work deals with Over-the-air (OTA) firmware updates. This paper consist of a theoretical explanation of attack vectors and its corresponding counter measures which can be applied. During the end of this paper, a detailed hands-on example with a ESP32 is given.