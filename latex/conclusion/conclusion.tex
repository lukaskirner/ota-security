\section{Conclusion}
The attack vectors and the corresponding defense techniques are not specific to over-the-air updates, moreover they can be used in general, for example to apply updates via Ethernet. The provided prevention techniques assume the hardware works secure and reliable. For example if the onboard WiFi module has a security flaw the effect of the previously mentioned prevention techniques may have no effect or can be skipped entirely.

\bigskip
Generally, firmware updates come with a cost. Besides the additional development and maintenance costs, there are also effects on the device/hardware itself. One example is the flash size, which needs to be at least twice the firmware size to simultaneously store the currently running firmware and the newly downloaded firmware. Cryptographic operations like verifying the image signature do also increase the overall load of the device, resulting in less computing power for the actual tasks. It is possible to reduce the additional load coming from the OTA update tasks by using different techniques, tools or libraries like using more efficient cryptographic algorithms like ed25519 or doing general optimizations, but at the end it's an additional task which requires computation power. These are some extra costs which should be taken into account on doing secure firmware updates. In some countries in this world firmware updates are an obligation. For example Germany recently introduced an update obligation \cite{Koltzsch2021}.
